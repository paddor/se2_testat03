\documentclass[a4paper]{article}
\usepackage[utf8]{inputenc}
\usepackage[a4paper]{geometry}
\geometry{verbose, marginparwidth=15mm, marginparsep=3mm, tmargin=25mm}
\usepackage[ngerman]{babel}
\usepackage{graphicx}
\usepackage[hyphens]{url}
\usepackage[flushmargin,hang]{footmisc}
\usepackage{hyperref}
\usepackage[parfill]{parskip} % no idented first line of each paragraph
\usepackage{amssymb} % for \checkmark
\usepackage{enumitem} % for \begin{itemize}[label={...}]
\usepackage{tikz}
\usetikzlibrary{arrows,decorations.pathmorphing,backgrounds,fit,positioning,shapes.symbols,chains,shapes.geometric,shapes.arrows,calc}
\hypersetup{
	unicode=true,
	pdftitle={SE2: Testat 3},
	pdfsubject={Refactoring von countlines},
	pdfauthor={Patrik Wenger},
	pdfkeywords={Java} {SE2} {refactoring} {JUnit},
}

\usepackage{pdfpages}
\title{SE2: Testat 3}
\author{Patrik Wenger}

\usepackage{color}
\usepackage{listings}
\lstdefinestyle{customjava}{ % custom style for C++ listings
  belowcaptionskip=1\baselineskip,
  breaklines=true,
  frame=single,
  xleftmargin=\parindent,
  language=java,
  showstringspaces=false,
  basicstyle=\footnotesize\ttfamily,
  keywordstyle=\bfseries\color{green!40!black},
  commentstyle=\itshape\color{purple!40!black},
  identifierstyle=\color{blue},
  stringstyle=\color{orange},
}
\newcommand{\java}[1]{\lstinline[style=customjava]{#1}} % inline Java

\begin{document}
\maketitle
\tableofcontents

\section{Ausgangslage}
Die Software \emph{countlines} ist das ausgew\"ahlte Beispielprojekt um
verschiedene Codesmells mit Hilfe der unten benutzten Tools zu finden. Es kann
verschiedene Arten von Codezeilen wie eigentlicher Code (netto),
Kommentarzeilen, leere Zeilen, sowie auch die totale Anzahl der Zeilen in
Quellcodedateien berechnen.  Dabei wird unter anderem Quellcode der Sprachen C,
C++, Java, Pascal, SQL und PHP unterst\"utzt.\\

\section{Identifizierte Code Smells}
Ich werde mich hier auf die Klasse \emph{LineCounter} konzentrieren, da hier
die Kernfunktionalit\"at enthalten ist und ich das GUI v\"ollig irrelevant
finde (AWT ist ja selbst schon getestet).

\begin{enumerate}
	\item \emph{Long Method}

		Dieser Smell ist ziemlich offensichtlich, jedoch nicht
		unbedingt sinvoll aufzul\"osen, da die jeweiligen Methoden den
		Parsingalgorhitmus enthalten, der sich auf einige lokale
		Variablen st\"utzt, aber sonst eigentlich selfcontained ist. Da
		sehe ich keinen grossen Nutzen, diese Methoden zu verk\"urzen.
		Es enst\"unde durch die Notwendigkeit f\"ur neue Felder sowieso
		nur ein neuer Smell: \emph{Temporary Fields}.

	\item \emph{Large Class}

		\emph{LineCounter} ist mit mehr als 250 Zeilen ziemlich gross
		und beinhaltet, wie die weiteren Smells zeigen, definitiv zu
		viel Verantwortung.

	\item \emph{Data Clumps}

		Die vier Anzahlen von verschiedenartigen Linien (Total, Netto,
		leere und Kommentarzeilen) werden immer miteinander gez\"ahlt
		und im Fenster auch miteinander angezeigt. Es macht also Sinn,
		diese mit \emph{Extract Component} in eine eigene, simple
		Datenklasse zu verpacken. Dies erlaubt dann auch einige
		Vereinfachungen in \emph{FileListTableModel}.

	\item \emph{Temporary Field}

		Die \java{static} Felder in der Klasse \emph{LineCounter}
		enthalten bedeutungslose Zahlen bevor die Methode
		\emph{countLines} zum ersten Mal auferufen wird. Dies ist
		verwirrend und ein weiterer Hinweis dazu, dass
		\emph{LineCounter} unter dem Smell \emph{Large Class} leidet.

	\item \emph{Divergent Change}

		Sofern ein neuer Dateityp unterst\"utzt werden soll, muss dies
		jeweils in mehreren Methoden der Klasse \emph{LineCounter}
		erfolgen. Hier macht also \emph{Replace Type Code with
		Subclasses} Sinn.

	\item \emph{Feature Envy}

		In \emph{Main} wird die Liste von Default-Dateiendungen
		angegeben. Wegen der Tatsache, dass sich der Code in
		\emph{LineCounter} auch auf die Endungen der Dateinamen
		verl\"asst um die Programmiersprache zu bestimmen, k\"onnte man
		argumentieren, dass eine sinnvolle Liste von standardm\"assig
		unterst\"utzten Dateiendungen direkt in die Klasse
		\emph{LineCounter} geh\"ort.

		Da dies jedoch schon der sechste Code Smell ist und die
		Meinungen \"uber diese Liste von Default-Dateiendungen
		auseinander gehen k\"onnten, belasse ich diesen Smell so.
\end{enumerate}

Hinzu kommt noch, dass die Klasse \emph{LineCounter} nicht thread-safe ist.
Dies ist so, weil die Resultate in \java{static} Felder gespeichert werden. Es
w\"are sch\"on, wenn sie von mehreren Threads verwendet werden k\"onnte.

Des weiteren beinhaltet zum Beispiel die Methode \java{initCountingVariables}
auch die Reinitialisierung der Variable \java{atEOF}, was v\"ollig fehl am
Platz ist.

\section{Erste Unit Tests}
Die ersten Unit Tests f\"ur die Klasse \emph{LineCounter} wurden eingebaut,
ohne jeglichen bestehenden Code zu ver\"andern. Dazu wurden einige Quelldateien
(Java, SQL, C, Text) als Fixtures erstellt. Deren Pfade werden in den Testcases
angegeben und die Resultate zu den einzelnen Zeilenanzahlen werden isoliert
\"uberpr\"uft.

Siehe commit
9d89bb\footnote{\url{https://github.com/paddor/se2_testat03/commit/9d89bbfda73249aec310644e5085c9a595be1bf7}}.

\section{Refactorings}

\subsection{\emph{Data Clump} beseitigen}
Mit dem Commit
537b73\footnote{\url{https://github.com/paddor/se2_testat03/commit/537b735a12270c6d9c2dde20abc13e511970f149}}
wurde der Smell \emph{Data Clumps} beseitigt, indem die zusammengeh\"origen
Zeilenanzahlen (Total, Netto, leere und Kommentarzeilen) in die neue Klasse
\java{LineCounts} gepackt werden. Die Felder sind \java{final}. F\"ur den
Zugriff werden die drei Methoden \java{getLineCount()},
\java{getTotalLineCount()}, \java{getEmptyLineCount()} und
\java{getCommentLineCount()} verwendet, statt direkten Zugriff auf die Felder
zuzulassen, was wieder ein Code Smell w\"are.

\subsection{\emph{Large Class} beseitigen}
Mit dem Commit
b417a4\footnote{\url{https://github.com/paddor/se2_testat03/commit/b417a4fd6d611a4bccc4564daf9d479acc84ca34}}
wurde dann schliesslich \emph{Replace Type Code with Subclasses} angewendet und
die neuen Klassen \java{Counter}, \java{JavaCounter}, \java{SqlCounter} und
\java{SimpleCounter} enstanden. Neben der Verk\"urzung vereinfachte sich Code
auch dadurch, dass die ausgelagerte Funktionalit\"at nun in Instanzmethoden
liegt, und nicht mehr in \java{static} Methoden. Dies bedeutet, dass das
manuelle Reinitializieren der Z\"ahlvariablen (und auch von \java{atEOF})
eleganterweise entf\"allt.

\subsection{Thread Safety}
Mit dem Commit
1f8266\footnote{\url{https://github.com/paddor/se2_testat03/commit/1f8266626a6b1a4da49e448b331657f887347c80}} wurde dann schliesslich die Methode \java{LineCounter.countLines()} so ge\"andert, dass sie das \java{LineCounts}-Objekt zur\"uckgibt und der Clientcode in \java{FileListTableModel} dementsprechend angepasst.

Mit den Commits
18fc92\footnote{\url{https://github.com/paddor/se2_testat03/commit/18fc92c421dd3dcf92b51d84ba63afa8c62adec5}}
und
ccad56\footnote{\url{https://github.com/paddor/se2_testat03/commit/ccad561721a5674c7841ae5ccd8de04c7cae942f}}
wurde dann entg\"ultig auf ver\"anderbare, geteilte Daten in \java{static} Variablen
verzichtet um das ganze thread-safe zu machen.


\section{Performance-Garantie}
Viel mehr als im urspr\"unglichen Code passiert nicht. Es gibt ein wenig mehr
Zwischenobjekte, jedoch ist der Code jetzt thread-safe, was heisst, dass ein
Prozess, der diesen Code ausf\"uhrt, nun mehrere Prozessoren gleichzeitig
belegen kann, ohne dass es Raceconditions gibt. Somit skaliert der neue Code
nun besser, was bei heutigen Multicore-Architekturen wichtiger ist als
sequenzielle Performance (welche wahrscheinlich unter den neuen
Zwischenobjekten unwesentlich leidet).

Aber ehrlich gesagt kann man keine grossen Aussagen \"uber die Performance
machen, ohne zu profilen.

\end{document}
